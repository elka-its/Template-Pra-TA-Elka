\clearpage
\thispagestyle{romanstyle} % nomor halaman romawi

\begin{center}
{\bfseries \judulTA} \\[1cm]
\end{center}

\noindent
\begin{tabular}{l l l}
\textbf{Nama Mahasiswa/NRP} & \textbf{:} & \textbf{\namamhs}/\textbf{\nrpMhs} \\
\textbf{Departemen}         & \textbf{:} & \textbf{\depart} \\
\textbf{Dosen Pembimbing}   & \textbf{:} & \textbf{\pembimbingSatu} \\
                            &            & \textbf{\pembimbingDua}
\end{tabular}


\vspace{1cm}

\vspace{1cm}

\noindent {\bfseries Abstrak} % judul ABSTRAK rata kiri, tanpa inden
\par % akhiri baris judul suapaya bawahnya ga ikutan

\setlength{\parindent}{1.25cm} % panjang inden paragraf
Abstrak ditulis dalam satu paragraf tunggal, berisi tentang hal-hal yang akan dikerjakan pada pelaksanaan Tugas Akhir yang terdiri dari 200-300 kata. Abstrak merupakan ringkasan rencana penelitian/rencana rancangan/proyek, yang berisi jawaban atas pertanyaan, apa, mengapa, dan bagaimana penelitian/rancangan yang akan dilakukan. Dalam sebuah abstrak, biasanya dijelaskan secara singkat latar belakang masalah yang diteliti, tujuan penelitian, metode yang digunakan, hasil utama yang diperoleh, serta kesimpulan yang diambil dari hasil penelitian tersebut. Abstrak yang baik harus ditulis dengan jelas dan padat, mencerminkan seluruh isi karya tulis secara akurat, serta mampu menarik minat pembaca untuk mendalami lebih lanjut penelitian yang dilakukan. Sebagai elemen penting dari karya ilmiah, abstrak sering kali menjadi bagian yang pertama kali dilihat oleh pembaca, sehingga kualitas penulisannya sangat krusial. Penulisan kata kunci dimulai dari abjad yang paling awal dan apabila terdapat kata dalam bahasa asing harap dimiringkan.

\vspace{0.5cm}
\noindent
{\bfseries Kata kunci:} Abstrak, Menarik, Padat, Penting, Singkat
