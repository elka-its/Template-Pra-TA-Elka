\clearpage
\thispagestyle{romanstyle} % nomor halaman romawi

\begin{center}
{\bfseries \judulinggris} \\[1cm]
\end{center}

\noindent
\begin{tabular}{l l l}
\textbf{Student Name/NRP} & \textbf{:} & \textbf{\namamhs}/\textbf{\nrpMhs} \\
\textbf{Department}         & \textbf{:} & \textbf{\depart} \\
\textbf{Advisor}   & \textbf{:} & \textbf{\pembimbingSatu} \\
                   &            & \textbf{\pembimbingDua}
\end{tabular}


\vspace{1cm}

\noindent {\bfseries Abstrak} % judul ABSTRAK rata kiri, tanpa inden
\par % akhiri baris judul suapaya bawahnya ga ikutan
\setlength{\parindent}{1.25cm} % panjang inden paragraf
The abstract is written in a single paragraph, containing information about the work to be carried out in the Final Project, consisting of 200-300 words. The abstract is a summary of the research plan/design plan/project, which contains answers to the questions of what, why, and how the research/design will be carried out. In an abstract, the background of the problem being studied, the research objectives, the methods used, the main results obtained, and the conclusions drawn from the research results are usually explained briefly. A good abstract should be written clearly and concisely, accurately reflecting the entire content of the paper, and be able to attract readers' interest to explore the research further. As an important element of scientific work, the abstract is often the first part that readers see, so the quality of its writing is crucial. Keywords should be written in alphabetical order, and if there are words in a foreign language, they should be italicised.


\vspace{0.5cm}
\noindent
{\bfseries Keywords: Abstract, Interesting, Concise, Important, Brief} 
