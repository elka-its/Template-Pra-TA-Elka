\chapter{TINJAUAN PUSTAKA}

\section{Hasil Studi Terdahulu}
Hasil studi terdahulu merupakan penelitian/perancangan yang relevan yang pernah dilakukan oleh berbagai pihak, dan apabila memungkinkan bukan hasil pelaksanaan Tugas Akhir (TA) terdahulu, melainkan dari jurnal ilmiah, paten, atau laporan perancangan lainnya dari lembaga yang kredibel. Hasil studi terdahulu ditulis dalam bentuk parafrasae bukan menjiplak karya dari orang lain yang bisa dibuat dalam bentuk tabel seperti Tabel 2.1, atau dalam bentuk uraian (bukan tabel).
\captionsetup[table]{position=top,justification=centering}
\begin{table}[ht]
\caption{Hasil Studi Terdahulu}
\centering
\renewcommand{\arraystretch}{1.3} % biar tinggi baris agak lega
% Definisi kolom untuk justified text
\newcolumntype{Z}{>{\justifying\arraybackslash}X}
\begin{tabularx}{\textwidth}{|c|l|Z|}
\hline
\textbf{No} & \textbf{Penulis, tahun} & \textbf{Hasil Penelitian} \\ \hline
1 & Ghalindri, 2022 & 
Analisis perbandingan waktu dan biaya dilaksanakan pada pekerjaan pondasi yang membandingkan penggunaan spun pile dan bore pile pada Proyek Bangunan Penunjang Skybridge Proyek Stasiun Integrasi LRT Halim, Jakarta. Berdasarkan hasil penelitian didapatkan kebutuhan waktu pemancangan dengan menggunakan spun pile 19 hari lebih cepat daripada bore pile. Sedangkan dari aspek biaya diperoleh penghematan sebesar 60,96\% dari Rencana Anggaran Biaya (RAB) awal sebesar Rp. 3.259.073.514,00 menjadi Rp. 1.986.593.514,00. \\ \hline
2 & Annisa, 2022 & 
Analisis perbandingan waktu dan biaya dilaksanakan pada pekerjaan bekisting yang membandingkan penggunaan bekisting konvensional dan aluminium formwork pada proyek Trans Icon Surabaya. Dari hasil penelitian didapatkan selisih pekerjaan menggunakan kedua metode dari segi biaya sebesar Rp. 600.732.246,00 lebih murah dan dari segi waktu 73 hari lebih cepat dengan menggunakan bekisting aluminium formwork. \\ \hline
\end{tabularx}
\end{table}

\section{Perencanaan Struktur Sekunder (Contoh)}
\subsection{Perencanaan Tangga Baja (Contoh)}
Berdasarkan SNI 1729:2015, penampang yang mengalami tekuk lokal dikategorikan sebagai 
elemen nonlangsing penampang elemen langsing. Untuk profil elemen nonlangsing, rasio 
tebal-terhadap-lebar dari elemen tekan tidak boleh melebihi batas elastisitas kelangsingan 
untuk elemen non kompak $\lambda_r$. Apabila rasio tersebut melebihi $\lambda_r$, disebut 
penampang dengan elemen langsing. Parameter batas kelangsingan elemen nonkompak 
$\lambda_r$ dapat dihitung dengan persamaan \eqref{eq:rumus}.
\begin{equation}
    \lambda_r = 1.49 \sqrt{\frac{E}{F_y}}
    \label{eq:rumus}
\end{equation}
\noindent
Keterangan:
\begin{itemize}[label={},leftmargin=\parindent,itemsep=2pt,topsep=2pt]
  \item $\lambda_r$ = Parameter batas kelangsingan untuk elemen nonkompak
  \item $E$ = Modulus elastisitas baja = 200000 MPa
  \item $F_y$ = Tegangan leleh minimum yang disyaratkan (MPa)
\end{itemize}

\subsection{Perencanaan Plat Lantai (Contoh)}
Berdasarkan SNI 1729:2015, penampang Syang mengalami tekuk lokal dikategorikan sebagai elemen nonlangsing penampang elemen-langsing. Untuk profil elemen nonlangsing, rasio tebal-terhadap-lebar dari elemen tekan tidak boleh melebihi batas elastisitas kelangsingan untuk elemen non kompak λr. Apabila rasio tersebut melebihi λr, disebut penampang dengan elemen-langsing. Menurut \cite{alexandersadiku}, kapasitor terdiri dari dua plat konduktor. Sedangkan menurut \cite{varaprasad}, kapasitor mampu menyimpan muatan listrik.


